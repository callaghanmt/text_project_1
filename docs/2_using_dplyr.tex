\documentclass[]{article}
\usepackage{lmodern}
\usepackage{amssymb,amsmath}
\usepackage{ifxetex,ifluatex}
\usepackage{fixltx2e} % provides \textsubscript
\ifnum 0\ifxetex 1\fi\ifluatex 1\fi=0 % if pdftex
  \usepackage[T1]{fontenc}
  \usepackage[utf8]{inputenc}
\else % if luatex or xelatex
  \ifxetex
    \usepackage{mathspec}
  \else
    \usepackage{fontspec}
  \fi
  \defaultfontfeatures{Ligatures=TeX,Scale=MatchLowercase}
\fi
% use upquote if available, for straight quotes in verbatim environments
\IfFileExists{upquote.sty}{\usepackage{upquote}}{}
% use microtype if available
\IfFileExists{microtype.sty}{%
\usepackage[]{microtype}
\UseMicrotypeSet[protrusion]{basicmath} % disable protrusion for tt fonts
}{}
\PassOptionsToPackage{hyphens}{url} % url is loaded by hyperref
\usepackage[unicode=true]{hyperref}
\hypersetup{
            pdftitle={Manipulating, analyzing and exporting data with tidyverse},
            pdfauthor={Data Carpentry contributors},
            pdfborder={0 0 0},
            breaklinks=true}
\urlstyle{same}  % don't use monospace font for urls
\usepackage{graphicx,grffile}
\makeatletter
\def\maxwidth{\ifdim\Gin@nat@width>\linewidth\linewidth\else\Gin@nat@width\fi}
\def\maxheight{\ifdim\Gin@nat@height>\textheight\textheight\else\Gin@nat@height\fi}
\makeatother
% Scale images if necessary, so that they will not overflow the page
% margins by default, and it is still possible to overwrite the defaults
% using explicit options in \includegraphics[width, height, ...]{}
\setkeys{Gin}{width=\maxwidth,height=\maxheight,keepaspectratio}
\IfFileExists{parskip.sty}{%
\usepackage{parskip}
}{% else
\setlength{\parindent}{0pt}
\setlength{\parskip}{6pt plus 2pt minus 1pt}
}
\setlength{\emergencystretch}{3em}  % prevent overfull lines
\providecommand{\tightlist}{%
  \setlength{\itemsep}{0pt}\setlength{\parskip}{0pt}}
\setcounter{secnumdepth}{0}
% Redefines (sub)paragraphs to behave more like sections
\ifx\paragraph\undefined\else
\let\oldparagraph\paragraph
\renewcommand{\paragraph}[1]{\oldparagraph{#1}\mbox{}}
\fi
\ifx\subparagraph\undefined\else
\let\oldsubparagraph\subparagraph
\renewcommand{\subparagraph}[1]{\oldsubparagraph{#1}\mbox{}}
\fi

% set default figure placement to htbp
\makeatletter
\def\fps@figure{htbp}
\makeatother


\title{Manipulating, analyzing and exporting data with tidyverse}
\author{Data Carpentry contributors}
\date{}

\begin{document}
\maketitle

\texttt{\{r,\ echo=FALSE,\ purl=FALSE,\ message\ =\ FALSE\}\ source("setup.R")\ surveys\ \textless{}-\ read.csv("data/portal\_data\_joined.csv")\ suppressWarnings(surveys\$date\ \textless{}-\ lubridate::ymd(paste(surveys\$year,\ \ \ \ \ \ \ \ \ \ \ \ \ \ \ \ \ \ \ \ \ \ \ \ \ \ \ \ \ \ \ \ \ \ \ \ \ \ \ \ \ \ \ \ \ \ \ \ \ \ \ \ \ \ \ surveys\$month,\ \ \ \ \ \ \ \ \ \ \ \ \ \ \ \ \ \ \ \ \ \ \ \ \ \ \ \ \ \ \ \ \ \ \ \ \ \ \ \ \ \ \ \ \ \ \ \ \ \ \ \ \ \ \ surveys\$day,\ \ \ \ \ \ \ \ \ \ \ \ \ \ \ \ \ \ \ \ \ \ \ \ \ \ \ \ \ \ \ \ \ \ \ \ \ \ \ \ \ \ \ \ \ \ \ \ \ \ \ \ \ \ \ sep\ =\ "-")))}

\subsubsection{Manipulating and analyzing data with
dplyr}\label{manipulating-and-analyzing-data-with-dplyr}

\begin{center}\rule{0.5\linewidth}{\linethickness}\end{center}

\begin{quote}
\subsubsection{Learning Objectives}\label{learning-objectives}

\begin{itemize}
\tightlist
\item
  Describe the purpose of the \textbf{\texttt{dplyr}} and
  \textbf{\texttt{tidyr}} packages.
\item
  Select certain columns in a data frame with the
  \textbf{\texttt{dplyr}} function \texttt{select}.
\item
  Select certain rows in a data frame according to filtering conditions
  with the \textbf{\texttt{dplyr}} function \texttt{filter} .
\item
  Link the output of one \textbf{\texttt{dplyr}} function to the input
  of another function with the `pipe' operator
  \texttt{\%\textgreater{}\%}.
\item
  Add new columns to a data frame that are functions of existing columns
  with \texttt{mutate}.
\item
  Use the split-apply-combine concept for data analysis.
\item
  Use \texttt{summarize}, \texttt{group\_by}, and \texttt{count} to
  split a data frame into groups of observations, apply a summary
  statistics for each group, and then combine the results.
\item
  Describe the concept of a wide and a long table format and for which
  purpose those formats are useful.
\item
  Describe what key-value pairs are.
\item
  Reshape a data frame from long to wide format and back with the
  \texttt{spread} and \texttt{gather} commands from the
  \textbf{\texttt{tidyr}} package.
\item
  Export a data frame to a .csv file.
\end{itemize}
\end{quote}

\begin{center}\rule{0.5\linewidth}{\linethickness}\end{center}

\section{\texorpdfstring{Data Manipulation using \textbf{\texttt{dplyr}}
and
\textbf{\texttt{tidyr}}}{Data Manipulation using dplyr and tidyr}}\label{data-manipulation-using-dplyr-and-tidyr}

Bracket subsetting is handy, but it can be cumbersome and difficult to
read, especially for complicated operations. Enter
\textbf{\texttt{dplyr}}. \textbf{\texttt{dplyr}} is a package for making
tabular data manipulation easier. It pairs nicely with
\textbf{\texttt{tidyr}} which enables you to swiftly convert between
different data formats for plotting and analysis.

Packages in R are basically sets of additional functions that let you do
more stuff. The functions we've been using so far, like \texttt{str()}
or \texttt{data.frame()}, come built into R; packages give you access to
more of them. Before you use a package for the first time you need to
install it on your machine, and then you should import it in every
subsequent R session when you need it. You should already have installed
the \textbf{\texttt{tidyverse}} package. This is an ``umbrella-package''
that installs several packages useful for data analysis which work
together well such as \textbf{\texttt{tidyr}}, \textbf{\texttt{dplyr}},
\textbf{\texttt{ggplot2}}, \textbf{\texttt{tibble}}, etc.

The \textbf{\texttt{tidyverse}} package tries to address 3 common issues
that arise when doing data analysis with some of functions that come
with R:

\begin{enumerate}
\def\labelenumi{\arabic{enumi}.}
\tightlist
\item
  The results from a base R function sometimes depend on the type of
  data.
\item
  Using R expressions in a non standard way, which can be confusing for
  new learners.
\item
  Hidden arguments, having default operations that new learners are not
  aware of.
\end{enumerate}

We have seen in our previous lesson that when building or importing a
data frame, the columns that contain characters (i.e., text) are coerced
(=converted) into the \texttt{factor} data type. We had to set
\textbf{\texttt{stringsAsFactors}} to \textbf{\texttt{FALSE}} to avoid
this hidden argument to convert our data type.

This time will use the \textbf{\texttt{tidyverse}} package to read the
data and avoid having to set \textbf{\texttt{stringsAsFactors}} to
\textbf{\texttt{FALSE}}

To load the package type:

\texttt{\{r,\ message\ =\ FALSE,\ purl\ =\ FALSE\}\ \#\#\ load\ the\ tidyverse\ packages,\ incl.\ dplyr\ library("tidyverse")}

\subsection{\texorpdfstring{What are \textbf{\texttt{dplyr}} and
\textbf{\texttt{tidyr}}?}{What are dplyr and tidyr?}}\label{what-are-dplyr-and-tidyr}

The package \textbf{\texttt{dplyr}} provides easy tools for the most
common data manipulation tasks. It is built to work directly with data
frames, with many common tasks optimized by being written in a compiled
language (C++). An additional feature is the ability to work directly
with data stored in an external database. The benefits of doing this are
that the data can be managed natively in a relational database, queries
can be conducted on that database, and only the results of the query are
returned.

This addresses a common problem with R in that all operations are
conducted in-memory and thus the amount of data you can work with is
limited by available memory. The database connections essentially remove
that limitation in that you can connect to a database of many hundreds
of GB, conduct queries on it directly, and pull back into R only what
you need for analysis.

The package \textbf{\texttt{tidyr}} addresses the common problem of
wanting to reshape your data for plotting and use by different R
functions. Sometimes we want data sets where we have one row per
measurement. Sometimes we want a data frame where each measurement type
has its own column, and rows are instead more aggregated groups - like
plots or aquaria. Moving back and forth between these formats is
nontrivial, and \textbf{\texttt{tidyr}} gives you tools for this and
more sophisticated data manipulation.

To learn more about \textbf{\texttt{dplyr}} and \textbf{\texttt{tidyr}}
after the workshop, you may want to check out this
\href{https://github.com/rstudio/cheatsheets/raw/master/data-transformation.pdf}{handy
data transformation with \textbf{\texttt{dplyr}} cheatsheet} and this
\href{https://github.com/rstudio/cheatsheets/raw/master/data-import.pdf}{one
about \textbf{\texttt{tidyr}}}.

We'll read in our data using the \texttt{read\_csv()} function, from the
tidyverse package \textbf{\texttt{readr}}, instead of
\texttt{read.csv()}.

```\{r, results = `hide', purl = FALSE\} surveys \textless{}-
read\_csv(``data/portal\_data\_joined.csv'')

\subsection{inspect the data}\label{inspect-the-data}

str(surveys)

\subsection{preview the data}\label{preview-the-data}

\section{View(surveys)}\label{viewsurveys}

```

Notice that the class of the data is now \texttt{tbl\_df}

This is referred to as a ``tibble''. Tibbles are data frames, but they
tweak some of the old behaviors of data frames. The data structure is
very similar to a data frame. For our purposes the only differences are
that:

\begin{enumerate}
\def\labelenumi{\arabic{enumi}.}
\tightlist
\item
  In addition to displaying the data type of each column under its name,
  it only prints the first few rows of data and only as many columns as
  fit on one screen.
\item
  Columns of class \texttt{character} are never converted into factors.
\end{enumerate}

We're going to learn some of the most common \textbf{\texttt{dplyr}}
functions: - \texttt{select()}: subset columns - \texttt{filter()}:
subset rows on conditions - \texttt{mutate()}: create new columns by
using information from other columns - \texttt{group\_by()} and
\texttt{summarize()}: create summary statisitcs on grouped data -
\texttt{arrange()}: sort results - \texttt{count()}: count discrete
values

\subsection{Selecting columns and filtering
rows}\label{selecting-columns-and-filtering-rows}

To select columns of a data frame, use \texttt{select()}. The first
argument to this function is the data frame (\texttt{surveys}), and the
subsequent arguments are the columns to keep.

\texttt{\{r,\ results\ =\ \textquotesingle{}hide\textquotesingle{},\ purl\ =\ FALSE\}\ select(surveys,\ plot\_id,\ species\_id,\ weight)}

To choose rows based on a specific criteria, use \texttt{filter()}:

\texttt{\{r,\ purl\ =\ FALSE\}\ filter(surveys,\ year\ ==\ 1995)}

\subsection{Pipes}\label{pipes}

What if you want to select and filter at the same time? There are three
ways to do this: use intermediate steps, nested functions, or pipes.

With intermediate steps, you create a temporary data frame and use that
as input to the next function, like this:

\texttt{\{r,\ purl\ =\ FALSE\}\ surveys2\ \textless{}-\ filter(surveys,\ weight\ \textless{}\ 5)\ surveys\_sml\ \textless{}-\ select(surveys2,\ species\_id,\ sex,\ weight)}

This is readable, but can clutter up your workspace with lots of objects
that you have to name individually. With multiple steps, that can be
hard to keep track of.

You can also nest functions (i.e.~one function inside of another), like
this:

\texttt{\{r,\ purl\ =\ FALSE\}\ surveys\_sml\ \textless{}-\ select(filter(surveys,\ weight\ \textless{}\ 5),\ species\_id,\ sex,\ weight)}

This is handy, but can be difficult to read if too many functions are
nested, as R evaluates the expression from the inside out (in this case,
filtering, then selecting).

The last option, \emph{pipes}, are a recent addition to R. Pipes let you
take the output of one function and send it directly to the next, which
is useful when you need to do many things to the same dataset. Pipes in
R look like \texttt{\%\textgreater{}\%} and are made available via the
\textbf{\texttt{magrittr}} package, installed automatically with
\textbf{\texttt{dplyr}}. If you use RStudio, you can type the pipe with
Ctrl + Shift + M if you have a PC or Cmd + Shift + M if you have a Mac.

\texttt{\{r,\ purl\ =\ FALSE\}\ surveys\ \%\textgreater{}\%\ \ \ filter(weight\ \textless{}\ 5)\ \%\textgreater{}\%\ \ \ select(species\_id,\ sex,\ weight)}

In the above code, we use the pipe to send the \texttt{surveys} dataset
first through \texttt{filter()} to keep rows where \texttt{weight} is
less than 5, then through \texttt{select()} to keep only the
\texttt{species\_id}, \texttt{sex}, and \texttt{weight} columns. Since
\texttt{\%\textgreater{}\%} takes the object on its left and passes it
as the first argument to the function on its right, we don't need to
explicitly include the data frame as an argument to the
\texttt{filter()} and \texttt{select()} functions any more.

Some may find it helpful to read the pipe like the word ``then''. For
instance, in the above example, we took the data frame \texttt{surveys},
\emph{then} we \texttt{filter}ed for rows with
\texttt{weight\ \textless{}\ 5}, \emph{then} we \texttt{select}ed
columns \texttt{species\_id}, \texttt{sex}, and \texttt{weight}. The
\textbf{\texttt{dplyr}} functions by themselves are somewhat simple, but
by combining them into linear workflows with the pipe, we can accomplish
more complex manipulations of data frames.

If we want to create a new object with this smaller version of the data,
we can assign it a new name:

```\{r, purl = FALSE\} surveys\_sml \textless{}- surveys
\%\textgreater{}\% filter(weight \textless{} 5) \%\textgreater{}\%
select(species\_id, sex, weight)

surveys\_sml ```

Note that the final data frame is the leftmost part of this expression.

\begin{quote}
\subsubsection{Challenge}\label{challenge}

Using pipes, subset the \texttt{surveys} data to include individuals
collected before 1995 and retain only the columns \texttt{year},
\texttt{sex}, and \texttt{weight}.

\texttt{\{r,\ answer=TRUE,\ eval=FALSE,\ purl=FALSE\}\ surveys\ \%\textgreater{}\%\ \ \ \ \ filter(year\ \textless{}\ 1995)\ \%\textgreater{}\%\ \ \ \ \ select(year,\ sex,\ weight)}
\end{quote}

\texttt{\{r,\ eval=FALSE,\ purl=TRUE,\ echo=FALSE\}\ \#\#\ Pipes\ Challenge:\ \#\#\ \ Using\ pipes,\ subset\ the\ data\ to\ include\ individuals\ collected\ \#\#\ \ before\ 1995,\ and\ retain\ the\ columns\ \textasciigrave{}year\textasciigrave{},\ \textasciigrave{}sex\textasciigrave{},\ and\ \textasciigrave{}weight.\textasciigrave{}}

\subsubsection{Mutate}\label{mutate}

Frequently you'll want to create new columns based on the values in
existing columns, for example to do unit conversions, or to find the
ratio of values in two columns. For this we'll use \texttt{mutate()}.

To create a new column of weight in kg:

\texttt{\{r,\ purl\ =\ FALSE\}\ surveys\ \%\textgreater{}\%\ \ \ mutate(weight\_kg\ =\ weight\ /\ 1000)}

You can also create a second new column based on the first new column
within the same call of \texttt{mutate()}:

\texttt{\{r,\ purl\ =\ FALSE\}\ surveys\ \%\textgreater{}\%\ \ \ mutate(weight\_kg\ =\ weight\ /\ 1000,\ \ \ \ \ \ \ \ \ \ weight\_kg2\ =\ weight\_kg\ *\ 2)}

If this runs off your screen and you just want to see the first few
rows, you can use a pipe to view the \texttt{head()} of the data. (Pipes
work with non-\textbf{\texttt{dplyr}} functions, too, as long as the
\textbf{\texttt{dplyr}} or \texttt{magrittr} package is loaded).

\texttt{\{r,\ purl\ =\ FALSE\}\ surveys\ \%\textgreater{}\%\ \ \ mutate(weight\_kg\ =\ weight\ /\ 1000)\ \%\textgreater{}\%\ \ \ head()}

The first few rows of the output are full of \texttt{NA}s, so if we
wanted to remove those we could insert a \texttt{filter()} in the chain:

\texttt{\{r,\ purl\ =\ FALSE\}\ surveys\ \%\textgreater{}\%\ \ \ filter(!is.na(weight))\ \%\textgreater{}\%\ \ \ mutate(weight\_kg\ =\ weight\ /\ 1000)\ \%\textgreater{}\%\ \ \ head()}

\texttt{is.na()} is a function that determines whether something is an
\texttt{NA}. The \texttt{!} symbol negates the result, so we're asking
for every row where weight \emph{is not} an \texttt{NA}.

\begin{quote}
\subsubsection{Challenge}\label{challenge-1}

Create a new data frame from the \texttt{surveys} data that meets the
following criteria: contains only the \texttt{species\_id} column and a
new column called \texttt{hindfoot\_half} containing values that are
half the \texttt{hindfoot\_length} values. In this
\texttt{hindfoot\_half} column, there are no \texttt{NA}s and all values
are less than 30.

\textbf{Hint}: think about how the commands should be ordered to produce
this data frame!

\texttt{\{r,\ answer=TRUE,\ eval=FALSE,\ purl=FALSE\}\ surveys\_hindfoot\_half\ \textless{}-\ surveys\ \%\textgreater{}\%\ \ \ \ \ filter(!is.na(hindfoot\_length))\ \%\textgreater{}\%\ \ \ \ \ mutate(hindfoot\_half\ =\ hindfoot\_length\ /\ 2)\ \%\textgreater{}\%\ \ \ \ \ filter(hindfoot\_half\ \textless{}\ 30)\ \%\textgreater{}\%\ \ \ \ \ select(species\_id,\ hindfoot\_half)}
\end{quote}

``\texttt{\{r,\ eval=FALSE,\ purl=TRUE,\ echo=FALSE\}\ \#\#\ Mutate\ Challenge:\ \#\#\ \ Create\ a\ new\ data\ frame\ from\ the}surveys\texttt{data\ that\ meets\ the\ following\ \#\#\ \ criteria:\ contains\ only\ the}species\_id\texttt{column\ and\ a\ column\ that\ \#\#\ \ contains\ values\ that\ are\ half\ the}hindfoot\_length\texttt{values\ (e.g.\ a\ \#\#\ \ new\ column}hindfoot\_half\texttt{).\ In\ this}hindfoot\_half`
column, there are \#\# no NA values and all values are \textless{} 30.

\subsection{Hint: think about how the commands should be ordered to
produce this data
frame!}\label{hint-think-about-how-the-commands-should-be-ordered-to-produce-this-data-frame}

```

\subsubsection{Split-apply-combine data analysis and the summarize()
function}\label{split-apply-combine-data-analysis-and-the-summarize-function}

Many data analysis tasks can be approached using the
\emph{split-apply-combine} paradigm: split the data into groups, apply
some analysis to each group, and then combine the results.
\textbf{\texttt{dplyr}} makes this very easy through the use of the
\texttt{group\_by()} function.

\paragraph{\texorpdfstring{The \texttt{summarize()}
function}{The summarize() function}}\label{the-summarize-function}

\texttt{group\_by()} is often used together with \texttt{summarize()},
which collapses each group into a single-row summary of that group.
\texttt{group\_by()} takes as arguments the column names that contain
the \textbf{categorical} variables for which you want to calculate the
summary statistics. So to compute the mean \texttt{weight} by sex:

\texttt{\{r,\ purl\ =\ FALSE\}\ surveys\ \%\textgreater{}\%\ \ \ group\_by(sex)\ \%\textgreater{}\%\ \ \ summarize(mean\_weight\ =\ mean(weight,\ na.rm\ =\ TRUE))}

You may also have noticed that the output from these calls doesn't run
off the screen anymore. It's one of the advantages of \texttt{tbl\_df}
over data frame.

You can also group by multiple columns:

\texttt{\{r,\ purl\ =\ FALSE\}\ surveys\ \%\textgreater{}\%\ \ \ group\_by(sex,\ species\_id)\ \%\textgreater{}\%\ \ \ summarize(mean\_weight\ =\ mean(weight,\ na.rm\ =\ TRUE))}

When grouping both by \texttt{sex} and \texttt{species\_id}, the last
few rows are for individuals that escaped before their sex could be
determined and weighted. You may notice that the last column does not
contain \texttt{NA} but \texttt{NaN} (which refers to ``Not a Number'').
To avoid this, we can remove the missing values for weight before we
attempt to calculate the summary statistics on weight. Because the
missing values are removed first, we can omit \texttt{na.rm\ =\ TRUE}
when computing the mean:

\texttt{\{r,\ purl\ =\ FALSE\}\ surveys\ \%\textgreater{}\%\ \ \ filter(!is.na(weight))\ \%\textgreater{}\%\ \ \ group\_by(sex,\ species\_id)\ \%\textgreater{}\%\ \ \ summarize(mean\_weight\ =\ mean(weight))}

Here, again, the output from these calls doesn't run off the screen
anymore. If you want to display more data, you can use the
\texttt{print()} function at the end of your chain with the argument
\texttt{n} specifying the number of rows to display:

\texttt{\{r,\ purl\ =\ FALSE\}\ surveys\ \%\textgreater{}\%\ \ \ filter(!is.na(weight))\ \%\textgreater{}\%\ \ \ group\_by(sex,\ species\_id)\ \%\textgreater{}\%\ \ \ summarize(mean\_weight\ =\ mean(weight))\ \%\textgreater{}\%\ \ \ print(n\ =\ 15)}

Once the data are grouped, you can also summarize multiple variables at
the same time (and not necessarily on the same variable). For instance,
we could add a column indicating the minimum weight for each species for
each sex:

\texttt{\{r,\ purl\ =\ FALSE\}\ surveys\ \%\textgreater{}\%\ \ \ filter(!is.na(weight))\ \%\textgreater{}\%\ \ \ group\_by(sex,\ species\_id)\ \%\textgreater{}\%\ \ \ summarize(mean\_weight\ =\ mean(weight),\ \ \ \ \ \ \ \ \ \ \ \ \ min\_weight\ =\ min(weight))}

It is sometimes useful to rearrange the result of a query to inspect the
values. For instance, we can sort on \texttt{min\_weight} to put the
species with the small hindfoot length first:

\texttt{\{r,\ purl\ =\ FALSE\}\ surveys\ \%\textgreater{}\%\ \ \ filter(!is.na(weight))\ \%\textgreater{}\%\ \ \ group\_by(sex,\ species\_id)\ \%\textgreater{}\%\ \ \ summarize(mean\_weight\ =\ mean(weight),\ \ \ \ \ \ \ \ \ \ \ \ \ min\_weight\ =\ min(weight))\ \%\textgreater{}\%\ \ \ arrange(min\_weight)}

To sort in descending order, we need to add the \texttt{desc()}
function. If we want to sort the results by decreasing order of mean
weight:

\texttt{\{r,\ purl\ =\ FALSE\}\ surveys\ \%\textgreater{}\%\ \ \ filter(!is.na(weight))\ \%\textgreater{}\%\ \ \ group\_by(sex,\ species\_id)\ \%\textgreater{}\%\ \ \ summarize(mean\_weight\ =\ mean(weight),\ \ \ \ \ \ \ \ \ \ \ \ \ min\_weight\ =\ min(weight))\ \%\textgreater{}\%\ \ \ arrange(desc(mean\_weight))}

\paragraph{Counting}\label{counting}

When working with data, we often want to know the number of observations
found for each factor or combination of factors. For this task,
\textbf{\texttt{dplyr}} provides \texttt{count()}. For example, if we
wanted to count the number of rows of data for each sex, we would do:

\texttt{\{r,\ purl\ =\ FALSE\}\ surveys\ \%\textgreater{}\%\ \ \ \ \ count(sex)}

For convenience, \texttt{count()} provides the \texttt{sort} argument:

\texttt{\{r,\ purl\ =\ FALSE\}\ surveys\ \%\textgreater{}\%\ \ \ \ \ count(sex,\ sort\ =\ TRUE)}

\begin{quote}
\subsubsection{Challenge}\label{challenge-2}

\begin{enumerate}
\def\labelenumi{\arabic{enumi}.}
\tightlist
\item
  How many individuals were caught in each \texttt{plot\_type} surveyed?
\end{enumerate}

\texttt{\{r,\ answer=TRUE,\ purl=FALSE\}\ surveys\ \%\textgreater{}\%\ \ \ \ \ count(plot\_type)}

\begin{enumerate}
\def\labelenumi{\arabic{enumi}.}
\setcounter{enumi}{1}
\tightlist
\item
  Use \texttt{group\_by()} and \texttt{summarize()} to find the mean,
  min, and max hindfoot length for each species (using
  \texttt{species\_id}). Also add the number of observations (hint: see
  \texttt{?n}).
\end{enumerate}

\texttt{\{r,\ answer=TRUE,\ purl=FALSE\}\ surveys\ \%\textgreater{}\%\ \ \ \ \ filter(!is.na(hindfoot\_length))\ \%\textgreater{}\%\ \ \ \ \ group\_by(species\_id)\ \%\textgreater{}\%\ \ \ \ \ summarize(\ \ \ \ \ \ \ \ \ mean\_hindfoot\_length\ =\ mean(hindfoot\_length),\ \ \ \ \ \ \ \ \ min\_hindfoot\_length\ =\ min(hindfoot\_length),\ \ \ \ \ \ \ \ \ max\_hindfoot\_length\ =\ max(hindfoot\_length),\ \ \ \ \ \ \ \ \ n\ =\ n()\ \ \ \ \ )}

\begin{enumerate}
\def\labelenumi{\arabic{enumi}.}
\setcounter{enumi}{2}
\tightlist
\item
  What was the heaviest animal measured in each year? Return the columns
  \texttt{year}, \texttt{genus}, \texttt{species\_id}, and
  \texttt{weight}.
\end{enumerate}

\texttt{\{r,\ answer=TRUE,\ purl=FALSE\}\ surveys\ \%\textgreater{}\%\ \ \ \ \ filter(!is.na(weight))\ \%\textgreater{}\%\ \ \ \ \ group\_by(year)\ \%\textgreater{}\%\ \ \ \ \ filter(weight\ ==\ max(weight))\ \%\textgreater{}\%\ \ \ \ \ select(year,\ genus,\ species,\ weight)\ \%\textgreater{}\%\ \ \ \ \ arrange(year)}
\end{quote}

``\texttt{\{r,\ eval=FALSE,\ purl=TRUE,\ echo=FALSE\}\ \#\#\ Count\ Challenges:\ \#\#\ \ 1.\ How\ many\ individuals\ were\ caught\ in\ each}plot\_type`
surveyed?

\subsection{\texorpdfstring{2. Use \texttt{group\_by()} and
\texttt{summarize()} to find the mean, min, and
max}{2. Use group\_by() and summarize() to find the mean, min, and max}}\label{use-group_by-and-summarize-to-find-the-mean-min-and-max}

\subsection{\texorpdfstring{hindfoot length for each species (using
\texttt{species\_id}). Also add the number
of}{hindfoot length for each species (using species\_id). Also add the number of}}\label{hindfoot-length-for-each-species-using-species_id.-also-add-the-number-of}

\subsection{\texorpdfstring{observations (hint: see
\texttt{?n}).}{observations (hint: see ?n).}}\label{observations-hint-see-n.}

\subsection{3. What was the heaviest animal measured in each year?
Return
the}\label{what-was-the-heaviest-animal-measured-in-each-year-return-the}

\subsection{\texorpdfstring{columns \texttt{year}, \texttt{genus},
\texttt{species\_id}, and
\texttt{weight}.}{columns year, genus, species\_id, and weight.}}\label{columns-year-genus-species_id-and-weight.}

```

\subsubsection{Reshaping with gather and
spread}\label{reshaping-with-gather-and-spread}

In the
\href{http://www.datacarpentry.org/spreadsheet-ecology-lesson/01-format-data/}{spreadsheet
lesson}, we discussed how to structure our data leading to the four
rules defining a tidy dataset:

\begin{enumerate}
\def\labelenumi{\arabic{enumi}.}
\tightlist
\item
  Each variable has its own column
\item
  Each observation has its own row
\item
  Each value must have its own cell
\item
  Each type of observational unit forms a table
\end{enumerate}

Here we examine the fourth rule: Each type of observational unit forms a
table.

In \texttt{surveys} , the rows of \texttt{surveys} contain the values of
variables associated with each record (the unit), values such the weight
or sex of each animal associated with each record. What if instead of
comparing records, we wanted to compare the different mean weight of
each species between plots? (Ignoring \texttt{plot\_type} for
simplicity).

We'd need to create a new table where each row (the unit) is comprise of
values of variables associated with each plot. In practical terms this
means the values of the species in \texttt{genus} would become the names
of column variables and the cells would contain the values of the mean
weight observed on each plot.

Having created a new table, it is therefore straightforward to explore
the relationship between the weight of different species within, and
between, the plots. The key point here is that we are still following a
tidy data structure, but we have \textbf{reshaped} the data according to
the observations of interest: average species weight per plot instead of
recordings per date.

The opposite transformation would be to transform column names into
values of a variable.

We can do both these of transformations with two \texttt{tidyr}
functions, \texttt{spread()} and \texttt{gather()}.

\paragraph{Spreading}\label{spreading}

\texttt{spread()} takes three principal arguments:

\begin{enumerate}
\def\labelenumi{\arabic{enumi}.}
\tightlist
\item
  the data
\item
  the \emph{key} column variable whose values will become new column
  names.\\
\item
  the \emph{value} column variable whose values will fill the new column
  variables.
\end{enumerate}

Further arguments include \texttt{fill} which, if set, fills in missing
values with the value provided.

Let's use \texttt{spread()} to transform surveys to find the mean weight
of each species in each plot over the entire survey period. We use
\texttt{filter()}, \texttt{group\_by()} and \texttt{summarise()} to
filter our observations and variables of interest, and create a new
variable for the \texttt{mean\_weight}. We use the pipe as before too.

```\{r, purl=FALSE\} surveys\_gw \textless{}- surveys \%\textgreater{}\%
filter(!is.na(weight)) \%\textgreater{}\% group\_by(genus, plot\_id)
\%\textgreater{}\% summarize(mean\_weight = mean(weight))

str(surveys\_gw) ```

This yields \texttt{surveys\_gw} where the observations for each plot
are spread across multiple rows, 196 observations of 13 variables. Using
\texttt{spread()} to key on \texttt{genus} with values from
\texttt{mean\_weight} this becomes 24 observations of 11 variables, one
row for each plot. We again use pipes:

```\{r, purl=FALSE\} surveys\_spread \textless{}- surveys\_gw
\%\textgreater{}\% spread(key = genus, value = mean\_weight)

str(surveys\_spread) ```

\begin{figure}
\centering
\includegraphics{img/spread_data_R.png}
\caption{}
\end{figure}

We could now plot comparisons between the weight of species in different
plots, although we may wish to fill in the missing values first.

\texttt{\{r,\ purl=FALSE\}\ surveys\_gw\ \%\textgreater{}\%\ \ \ spread(genus,\ mean\_weight,\ fill\ =\ 0)\ \%\textgreater{}\%\ \ \ head()}

\paragraph{Gathering}\label{gathering}

The opposing situation could occur if we had been provided with data in
the form of \texttt{surveys\_spread}, where the genus names are column
names, but we wish to treat them as values of a genus variable instead.

In this situation we are gathering the column names and turning them
into a pair of new variables. One variable represents the column names
as values, and the other variable contains the values previously
associated with the column names.

\texttt{gather()} takes four principal arguments:

\begin{enumerate}
\def\labelenumi{\arabic{enumi}.}
\tightlist
\item
  the data
\item
  the \emph{key} column variable we wish to create from column names.
\item
  the \emph{values} column variable we wish to create and fill with
  values associated with the key.
\item
  the names of the columns we use to fill the key variable (or to drop).
\end{enumerate}

To recreate \texttt{surveys\_gw} from \texttt{surveys\_spread} we would
create a key called \texttt{genus} and value called
\texttt{mean\_weight} and use all columns except \texttt{plot\_id} for
the key variable. Here we drop \texttt{plot\_id} column with a minus
sign.

```\{r, purl=FALSE\} surveys\_gather \textless{}- surveys\_spread
\%\textgreater{}\% gather(key = genus, value = mean\_weight, -plot\_id)

str(surveys\_gather) ```

\begin{figure}
\centering
\includegraphics{img/gather_data_R.png}
\caption{}
\end{figure}

Note that now the \texttt{NA} genera are included in the re-gathered
format. Spreading and then gathering can be a useful way to balance out
a dataset so every replicate has the same composition.

We could also have used a specification for what columns to include.
This can be useful if you have a large number of identifying columns,
and it's easier to specify what to gather than what to leave alone. And
if the columns are in a row, we don't even need to list them all out -
just use the \texttt{:} operator!

\texttt{\{r,\ purl=FALSE\}\ surveys\_spread\ \%\textgreater{}\%\ \ \ gather(key\ =\ genus,\ value\ =\ mean\_weight,\ Baiomys:Spermophilus)\ \%\textgreater{}\%\ \ \ head()}

\begin{quote}
\subsubsection{Challenge}\label{challenge-3}

\begin{enumerate}
\def\labelenumi{\arabic{enumi}.}
\tightlist
\item
  Spread the \texttt{surveys} data frame with \texttt{year} as columns,
  \texttt{plot\_id} as rows, and the number of genera per plot as the
  values. You will need to summarize before reshaping, and use the
  function \texttt{n\_distinct()} to get the number of unique genera
  within a particular chunk of data. It's a powerful function! See
  \texttt{?n\_distinct} for more.
\end{enumerate}

```\{r, answer=TRUE, purl=FALSE\} rich\_time \textless{}- surveys
\%\textgreater{}\% group\_by(plot\_id, year) \%\textgreater{}\%
summarize(n\_genera = n\_distinct(genus)) \%\textgreater{}\%
spread(year, n\_genera)

head(rich\_time) ```

\begin{enumerate}
\def\labelenumi{\arabic{enumi}.}
\setcounter{enumi}{1}
\tightlist
\item
  Now take that data frame and \texttt{gather()} it again, so each row
  is a unique \texttt{plot\_id} by \texttt{year} combination.
\end{enumerate}

\texttt{\{r,\ answer=TRUE,\ purl=FALSE\}\ rich\_time\ \%\textgreater{}\%\ \ \ gather(year,\ n\_genera,\ -plot\_id)}

\begin{enumerate}
\def\labelenumi{\arabic{enumi}.}
\setcounter{enumi}{2}
\tightlist
\item
  The \texttt{surveys} data set has two measurement columns:
  \texttt{hindfoot\_length} and \texttt{weight}. This makes it difficult
  to do things like look at the relationship between mean values of each
  measurement per year in different plot types. Let's walk through a
  common solution for this type of problem. First, use \texttt{gather()}
  to create a dataset where we have a key column called
  \texttt{measurement} and a \texttt{value} column that takes on the
  value of either \texttt{hindfoot\_length} or \texttt{weight}.
  \emph{Hint}: You'll need to specify which columns are being gathered.
\end{enumerate}

\texttt{\{r,\ answer=TRUE,\ purl=FALSE\}\ surveys\_long\ \textless{}-\ surveys\ \%\textgreater{}\%\ \ \ gather(measurement,\ value,\ hindfoot\_length,\ weight)}

\begin{enumerate}
\def\labelenumi{\arabic{enumi}.}
\setcounter{enumi}{3}
\tightlist
\item
  With this new data set, calculate the average of each
  \texttt{measurement} in each \texttt{year} for each different
  \texttt{plot\_type}. Then \texttt{spread()} them into a data set with
  a column for \texttt{hindfoot\_length} and \texttt{weight}.
  \emph{Hint}: You only need to specify the key and value columns for
  \texttt{spread()}.
\end{enumerate}

\texttt{\{r,\ answer=TRUE,\ purl=FALSE\}\ surveys\_long\ \%\textgreater{}\%\ \ \ group\_by(year,\ measurement,\ plot\_type)\ \%\textgreater{}\%\ \ \ summarize(mean\_value\ =\ mean(value,\ na.rm=TRUE))\ \%\textgreater{}\%\ \ \ spread(measurement,\ mean\_value)}
\end{quote}

```\{r, eval=FALSE, purl=TRUE, echo=FALSE\} \#\# Reshaping challenges

\subsection{\texorpdfstring{1. Make a wide data frame with \texttt{year}
as columns,
\texttt{plot\_id\textasciigrave{}\textasciigrave{}\ as\ rows,\ and\ where\ the\ values\ are\ the\ number\ of\ genera\ per\ plot.\ You\ will\ need\ to\ summarize\ before\ reshaping,\ and\ use\ the\ function}n\_distinct\texttt{to\ get\ the\ number\ of\ unique\ genera\ within\ a\ chunk\ of\ data.\ It\textquotesingle{}s\ a\ powerful\ function!\ See}?n\_distinct`
for
more.}{1. Make a wide data frame with year as columns, plot\_id`` as rows, and where the values are the number of genera per plot. You will need to summarize before reshaping, and use the functionn\_distinctto get the number of unique genera within a chunk of data. It's a powerful function! See?n\_distinct` for more.}}\label{make-a-wide-data-frame-with-year-as-columns-plot_id-as-rows-and-where-the-values-are-the-number-of-genera-per-plot.-you-will-need-to-summarize-before-reshaping-and-use-the-functionn_distinctto-get-the-number-of-unique-genera-within-a-chunk-of-data.-its-a-powerful-function-seen_distinct-for-more.}

\subsection{\texorpdfstring{2. Now take that data frame, and make it
long again, so each row is a unique \texttt{plot\_id} \texttt{year}
combination}{2. Now take that data frame, and make it long again, so each row is a unique plot\_id year combination}}\label{now-take-that-data-frame-and-make-it-long-again-so-each-row-is-a-unique-plot_id-year-combination}

\subsection{\texorpdfstring{3. The \texttt{surveys} data set is not
truly wide or long because there are two columns of measurement -
\texttt{hindfoot\_length} and \texttt{weight}. This makes it difficult
to do things like look at the relationship between mean values of each
measurement per year in different plot types. Let's walk through a
common solution for this type of problem. First, use \texttt{gather} to
create a truly long dataset where we have a key column called
\texttt{measurement} and a \texttt{value} column that takes on the value
of either \texttt{hindfoot\_length} or \texttt{weight}. Hint: You'll
need to specify which columns are being
gathered.}{3. The surveys data set is not truly wide or long because there are two columns of measurement - hindfoot\_length and weight. This makes it difficult to do things like look at the relationship between mean values of each measurement per year in different plot types. Let's walk through a common solution for this type of problem. First, use gather to create a truly long dataset where we have a key column called measurement and a value column that takes on the value of either hindfoot\_length or weight. Hint: You'll need to specify which columns are being gathered.}}\label{the-surveys-data-set-is-not-truly-wide-or-long-because-there-are-two-columns-of-measurement---hindfoot_length-and-weight.-this-makes-it-difficult-to-do-things-like-look-at-the-relationship-between-mean-values-of-each-measurement-per-year-in-different-plot-types.-lets-walk-through-a-common-solution-for-this-type-of-problem.-first-use-gather-to-create-a-truly-long-dataset-where-we-have-a-key-column-called-measurement-and-a-value-column-that-takes-on-the-value-of-either-hindfoot_length-or-weight.-hint-youll-need-to-specify-which-columns-are-being-gathered.}

\subsection{\texorpdfstring{4. With this new truly long data set,
calculate the average of each \texttt{measurement} in each \texttt{year}
for each different \texttt{plot\_type}. Then \texttt{spread} them into a
wide data set with a column for \texttt{hindfoot\_length} and
\texttt{weight}. Hint: Remember, you only need to specify the key and
value columns for
\texttt{spread}.}{4. With this new truly long data set, calculate the average of each measurement in each year for each different plot\_type. Then spread them into a wide data set with a column for hindfoot\_length and weight. Hint: Remember, you only need to specify the key and value columns for spread.}}\label{with-this-new-truly-long-data-set-calculate-the-average-of-each-measurement-in-each-year-for-each-different-plot_type.-then-spread-them-into-a-wide-data-set-with-a-column-for-hindfoot_length-and-weight.-hint-remember-you-only-need-to-specify-the-key-and-value-columns-for-spread.}

```

\section{Exporting data}\label{exporting-data}

Now that you have learned how to use \textbf{\texttt{dplyr}} to extract
information from or summarize your raw data, you may want to export
these new data sets to share them with your collaborators or for
archival.

Similar to the \texttt{read\_csv()} function used for reading CSV files
into R, there is a \texttt{write\_csv()} function that generates CSV
files from data frames.

Before using \texttt{write\_csv()}, we are going to create a new folder,
\texttt{data\_output}, in our working directory that will store this
generated dataset. We don't want to write generated datasets in the same
directory as our raw data. It's good practice to keep them separate. The
\texttt{data} folder should only contain the raw, unaltered data, and
should be left alone to make sure we don't delete or modify it. In
contrast, our script will generate the contents of the
\texttt{data\_output} directory, so even if the files it contains are
deleted, we can always re-generate them.

In preparation for our next lesson on plotting, we are going to prepare
a cleaned up version of the data set that doesn't include any missing
data.

Let's start by removing observations for which the \texttt{species\_id}
is missing. In this data set, the missing species are represented by an
empty string and not an \texttt{NA}. Let's also remove observations for
which \texttt{weight} and the \texttt{hindfoot\_length} are missing.
This data set should also only contain observations of animals for which
the sex has been determined:

\texttt{\{r,\ purl=FALSE\}\ surveys\_complete\ \textless{}-\ surveys\ \%\textgreater{}\%\ \ \ filter(!is.na(weight),\ \ \ \ \ \ \ \ \ \ \ \#\ remove\ missing\ weight\ \ \ \ \ \ \ \ \ \ !is.na(hindfoot\_length),\ \ \#\ remove\ missing\ hindfoot\_length\ \ \ \ \ \ \ \ \ \ !is.na(sex))\ \ \ \ \ \ \ \ \ \ \ \ \ \ \ \ \#\ remove\ missing\ sex}

Because we are interested in plotting how species abundances have
changed through time, we are also going to remove observations for rare
species (i.e., that have been observed less than 50 times). We will do
this in two steps: first we are going to create a data set that counts
how often each species has been observed, and filter out the rare
species; then, we will extract only the observations for these more
common species:

```\{r, purl=FALSE\} \#\# Extract the most common species\_id
species\_counts \textless{}- surveys\_complete \%\textgreater{}\%
count(species\_id) \%\textgreater{}\% filter(n \textgreater{}= 50)

\subsection{Only keep the most common
species}\label{only-keep-the-most-common-species}

surveys\_complete \textless{}- surveys\_complete \%\textgreater{}\%
filter(species\_id \%in\% species\_counts\$species\_id) ```

``\texttt{\{r,\ eval=FALSE,\ purl=TRUE,\ echo=FALSE\}\ \#\#\#\ Create\ the\ dataset\ for\ exporting:\ \#\#\ \ Start\ by\ removing\ observations\ for\ which\ the}species\_id\texttt{,}weight\texttt{,\ \#\#}hindfoot\_length\texttt{,\ or}sex`
data are missing: surveys\_complete \textless{}- surveys
\%\textgreater{}\% filter(species\_id != ``'', \# remove missing
species\_id !is.na(weight), \# remove missing weight
!is.na(hindfoot\_length), \# remove missing hindfoot\_length sex !=
``'') \# remove missing sex

\subsection{Now remove rare species in two steps. First, make a list of
species
which}\label{now-remove-rare-species-in-two-steps.-first-make-a-list-of-species-which}

\subsection{appear at least 50 times in our
dataset:}\label{appear-at-least-50-times-in-our-dataset}

species\_counts \textless{}- surveys\_complete \%\textgreater{}\%
count(species\_id) \%\textgreater{}\% filter(n \textgreater{}= 50)
\%\textgreater{}\% select(species\_id)

\subsection{Second, keep only those
species:}\label{second-keep-only-those-species}

surveys\_complete \textless{}- surveys\_complete \%\textgreater{}\%
filter(species\_id \%in\% species\_counts\$species\_id) ```

To make sure that everyone has the same data set, check that
\texttt{surveys\_complete} has \texttt{r\ nrow(surveys\_complete)} rows
and \texttt{r\ ncol(surveys\_complete)} columns by typing
\texttt{dim(surveys\_complete)}.

Now that our data set is ready, we can save it as a CSV file in our
\texttt{data\_output} folder.

\texttt{\{r,\ purl=FALSE,\ eval=FALSE\}\ write\_csv(surveys\_complete,\ path\ =\ "data\_output/surveys\_complete.csv")}

\texttt{\{r,\ purl=FALSE,\ eval=TRUE,\ echo=FALSE\}\ if\ (!dir.exists("data\_output"))\ dir.create("data\_output")\ write\_csv(surveys\_complete,\ path\ =\ "data\_output/surveys\_complete.csv")}

\texttt{\{r,\ child="\_page\_built\_on.Rmd"\}}

\end{document}
